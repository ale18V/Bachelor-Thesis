\section{Blockchain Classification}\label{sec:blockchain-classification}
\subsection{Public vs Private Blockchains}
Public blockchains are open to anyone, providing a high level of decentralization and
transparency. However, this also implies higher costs associated with blockchain usage and also
increases the risk of incurring into malicious actors, as anyone can participate without prior verification.

In contrast, private blockchains restrict access to authorized participants, allowing more control and accountability.
Private blockchains are often preferable for enterprise applications, where security and control
are paramount, and accountability can be enforced among known participants. Being specialized for a dedicated task,
the costs associated with blockchain usage are also lower.

\subsection{Consensus Mechanisms}\label{sec:consensus}
Consensus mechanisms are algorithms which coordinate the participants of the chain towards reaching agreement
on the global state of the network. Different consensus mechanisms have varying impacts on performance,
energy consumption, and overall network security.

Proof of Work (PoW) was the first consensus mechanisms to be adopted. In PoW, miners
solve cryptographic puzzles to validate transactions, which requires significant computational power and
energy. While PoW ensures security, it is energy-intensive and can cause delays due to mining time, as well
as increase the risk of blockchain forks. These factors limit its scalability and efficiency for real-time applications.

Proof of Stake (PoS) addresses some of PoW's shortcomings by selecting validators based on the amount of
cryptocurrency they are willing to stake. PoS reduces energy consumption and lowers forking risks since
validators are financially incentivized to act honestly. Additionally, PoS networks tend to be more scalable
and faster.

Consortium-based mechanisms, often used in permissioned blockchains, involve a select group of trusted
entities validating transactions. These systems are more energy-efficient than PoW and offer faster
transaction processing due to fewer validators. This mechanism is the most energy efficient and scalable out
of the three however it comes with the downside of being more centralized.

\subsection{Smart Contract-Based Platforms}\label{sec:smart-contracts}
While custom blockchains allow for greater control and optimization specific to a use case, smart contracts
provide a way to leverage existing, widely adopted blockchain platforms like Ethereum. Smart contracts are
self-executing contracts with the terms directly written into code. This makes them ideal for automating
tasks such as managing transactions and coordinating activities in decentralized applications. They allow
federated learning frameworks to be easily deployed across different blockchains, as long as those platforms
support similar contract functionality. This compatibility enables easier integration with established
blockchain networks and reduces the need for building infrastructure from scratch.