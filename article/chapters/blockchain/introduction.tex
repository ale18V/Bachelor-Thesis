\section{Introduction to Blockchain}
Nowadays blockchain has become a buzzword in the tech industry and it gets thrown around in many context just
to generate hype. Many people view it as something mysterious and cryptic.
However the core idea behind the technology is very simple: a blockchain is just a linked list of blocks
where each new block can only be appended at the end of the list. The data can represent anything, from
financial transactions to medical records: as a matter of fact, the first use case of
blockchain was timestamping documents.

The definition we just gave is very simple but intuitive. However when we talk about blockchain we usually
refer to the ecosystem and philosophy surround it as well. A more precise definition is the following:
\begin{quote}
  \it A blockchain is a decentralized and distributed
  digital ledger that records \textit{transactions} across multiple
  computers in a way that ensures the security, transparency, and immutability of the data.
\end{quote}

We use to call transactions the events that are recorded in the chain even though they may not necessarily be
associated with finance. Each transaction on a blockchain is grouped into a \textbf{block} which is linked to
previous blocks, forming the append-only linked list.

The core idea of a blockchain is that once a block is validated and added to the chain, it is very hard to
alter or delete it, making the blockchain highly resistant to tampering and fraud.
This is made possible via a consensus mechanism \ref{sec:consensus}, which is a process that multiple nodes
(computers) follow to agree on the validity of blocks before adding them to the chain.

Blockchain technology enables secure, peer-to-peer transactions without the need for a central authority,
such as a bank or government, because each participant in the network holds a copy of the entire blockchain.
This decentralized structure is what ensures data integrity, as any changes to a transaction would require
the consensus of the majority of the network.
The transparency of the blockchain means that all participants can see the transaction history, making it a
trusted source for verifying information.

Originally developed as the technology underpinning cryptocurrencies like Bitcoin, blockchain has since
expanded to numerous applications beyond finance, including supply chain management, voting systems, and healthcare.
As we will see in \ref{sec:smart-contracts}, smart contracts — programmable agreements that automatically execute
when predefined conditions are met — are another innovative use of blockchain, as they enable automated and
enforceable contracts without intermediaries.

Blockchain technology is still a field of ongoing research and, while the core principles have to remain
consistent with what we stated above, there are many diverse implementation paths that can be taken when
building a network based on a blockchain.
In order to provide a foundational understanding of state of the technology and its limitations, we will
explore the features and characteristics of the biggest and first implemented public blockchain: Bitcoin.
This analysis will also be beneficial to the understanding of the project implementation as we will see in
\ref{chap:implementation}.
