\phantomsection
\chapter*{Conclusion}\label{chap:conclusion}\addcontentsline{toc}{chapter}{Conclusion}
After all the topics discussed it's time to sum up and draw the conclusions.
This thesis has explored the integration of Federated Learning and Blockchain, focusing on
their potential to address critical challenges in privacy, security, and decentralization.
Federated Learning facilitates collaborative model training without centralizing data, while Blockchain
enhances trust through
its decentralized and immutable ledger.

\subsection*{Key Findings}
\begin{itemize}
  \item \textbf{Enhanced Security and Privacy}: Combining Federated Learning with Blockchain mitigates
    reliance on a central
    server, addressing privacy concerns and trust issues inherent in traditional FL systems.
  \item \textbf{Improved Transparency and Trust}: Blockchain's transparency ensures model integrity, while
    smart contracts and consensus mechanisms provide accountability.
  \item \textbf{Scalability Challenges}: The integration introduces significant computational and
    communication overhead, especially in environments with constrained resources.
\end{itemize}

\subsection*{Contributions}
\begin{itemize}
  \item A Blockchain-augmented FL framework was proposed, demonstrating its feasibility through both
    theoretical analysis and practical implementation.
  \item Comparative evaluations against vanilla federated learning revealed that using blockchain is a valid
    choice as it does not significantly impact the performance of the learning task.
\end{itemize}

\subsection*{Remarks}
In our experiment we have dealt with a very simple dataset of limited size and complexity. Furthermore we
only used ten nodes and the communication happened not even on a local network but via the loopback address
of the machine. While the results are promising, they should be taken with a grain of salt as they may not be
representative of
real-world scenarios.

\subsection*{Future Directions}
When discussing the challenges we put at stake many problems at once but only addressed few of them in our
implementation.

Furthermore we have not dealt with other practical limitations. For instance if we're interested in deploying
a real-world Blockchain FL system we would have to consider the fact that most user-owned devices have
access to the internet only via NAT and therefore are not able to accept inbound connections. This would make
the idea of using a peer to peer network unfeasible.

Another very important problem we have not dealt with is storage scalability. As we have highlited in
\ref{subsec:challenges-blockchain-fl}
storing all the updates of the model on the blockchain can outgrow storage capabilities of FL participants very fast.

Our conclusion is that the integration of FL and Blockchain is promising but remains in its early stages.
