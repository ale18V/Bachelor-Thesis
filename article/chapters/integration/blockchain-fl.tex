\section{Introduction}\label{sec:blockchain-fl}
In this section we would like to explore the possibility of integrating blockchain technology with federated learning.
As we mentioned in \ref{sec:challenges-vanilla-fl} centralization and lack of incentives are big problems in
federated learning. In \ref{subsec:advantages-blockchain-fl} we will discuss how implementing it on top of a
blockchain network helps solve these problems. In \ref{subsec:challenges-blockchain-fl} we will instead discuss
the novel challenges and complexities introduced by mixing these two technologies.

\subsection{Architecture}
First of all let's discuss the workflow of the system and the possible implementation choices.
Note that this is not a client-server model anymore. Thus the parties to which we referred as
\textit{clients} in \ref{sec:vanilla-fl} will now be referred to as \textit{nodes} or \textit{devices}:
\begin{enumerate}
  \item The global model does not need to be pulled from a central server.
    Each node can retrieve it by inspecting the blockchain.
  \item The nodes behaves as in \ref{sec:clients-and-datasets}: they train the local model on their own data.
  \item Once training is finished, the local update is broadcasted on the peer to peer network.
  \item The update is put inside a block and becomes part of the blockchain.
  \item When a new block is broadcasted to the network the clients update their model accordingly.
\end{enumerate}

There are many implementation details that can be explored in more detail about this architecture and that's
exactly what will be done now.

\paragraph{Global Model} The global model will always be retrievable from the blockchain as all updates are
recorded immutably.
Thus one could reconstruct the global model by simply inspecting the blockchain history starting from the genesis block.
In practice clients would just store it somewhere and gradually update it as new blocks are received.
Clearly it is unreasonable to expect nodes to be online continuously, as the devices participating in
federated learning often lack this capability \ref{sec:vanilla-fl}.
As a matter of fact when nodes come back online they can query other nodes in the network for the blocks
they're missing and use those to perform the necessary updates.

\paragraph{Block creation} This point was intentionally very vague in the presented overview.
This is because it's heavily implementation-dependant: in a PoW blockchain the responsibility of
creating new blocks and pooling updates is left to miners; in a PoS blockchain, the responsibility would fall
to an elected node.
Furthermore, should the updates be validated? And if so who should be responsible for doing so?
Should the updates be stored singularly inside the block or only their aggregated value?
Those question require careful analysis of the task at hand so it is not possible to provide a clear-cut
answer beforehand.

\subsection{Advantages of blockchain integration}\label{subsec:advantages-blockchain-fl}
\subsubsection{Decentralized Aggregation}
Decentralized aggregation as described in the last section, removes the need for a central server to handle
model updates. This nullifies the concerns which were reaised in \ref{sec:fl-centralization}: there is no
single point of
failure anymore and the nodes do not need to trust anyone as they can verify the updates independently.

\subsubsection{Incentivizing Participation}
Implementing an incentive mechanism is not a possibility exclusive to blockchain.
It is trivial to implement one in vanilla federated learning too:
simply let every client have a balance on the central server and have the latter update it when contributions
are received.

However this approach comes with the problems of centralization that have already been explained: clients
have no guarantee that the server will distribute rewards fairly.

On the contrary a blockchain can be used to develop a transparent incentive mechanism, agreed upon by all
nodes via the consensus algorithm such as demonstrated in \cite{FlwrBC}.

\subsubsection{Reputation and Accountability}
Permanently recording contributions of each client in a transparent manner allows the development of a trustless
reputation system, where clients that consistently provide high-quality updates can build
credibility.

Conversely, malicious or unreliable contributors can be identified and penalized,
enhancing model robustness and accountability.
The reputation of nodes could also be integrated with the reward mechanism such that higher quality contributors
are favored with more rewards whereas malicious contributors are penalized.

\subsection{Challenges of Blockchain-Integrated FL}\label{subsec:challenges-blockchain-fl}
\subsubsection{Model Size and Scalability}
Integrating blockchain with federated learning introduces scalability challenges, particularly
concerning model size. Storing large models or frequent updates directly on the blockchain is
impractical, as it would overwhelm the network.

Here are some possible (non-exclusive) mitigations to this problem:
\begin{itemize}
  \item Only store references to model updates on chain and store the data somewhere else.
    This doesn't necessarily mean that data should be stored centrally as a distributed storage system may be
    used such as IPFS.
    This would reduce the load on the blockchain but would introduce new problems such as data integrity and security.
  \item Have some nodes specialized in storage via Proof of Storage. This is the solution that is proposed in
    \cite{CoinAI}.
  \item Keep storing the updates on-chain and eventually prune old blocks when the storage amount exceeds the
    chain capabilities.
    This approach introduces non-trivial technical problems such as ensuring that all nodes consent and
    become aware of the pruning.
  \item Split nodes into lightweight and full nodes such as in \ref{sec:bitcoin-nodes}.
  \item Use model compression to reduce data size.
\end{itemize}
As discussed in \ref{sec:challenges-vanilla-fl} also here it is vital to use a \textit{relaxed} block
frequency and perform more computation locally.
In any case, even after performing the due optimizations, integrating blockchain in a federated learning task
may not always be possible if we're dealing with devices with very limited storage capabilities.

\subsubsection{Increased Overhead}
Blockchain networks inherently introduce latency due to the need for consensus and block validation.
This latency can slow down the learning task, as each model update must wait for
confirmation on the blockchain before it is available for aggregation. As a result,
blockchain-integrated federated learning may struggle to meet real-time or near-real-time
model update requirements.

Another situation which would introduce a huge overhead would be forking.
Forks are common in public blockchain networks, particularly those using
Proof of Work (PoW) consensus. When a fork occurs, different parts of the network temporarily
diverge, which means some nodes will train local updates on a model whereas some other nodes will do so on a
different modal.
Once the fork is resolved one of the two divergenct models will have to be discarded, leading to wasted computation.
Thus minimizing forking risks is essential, potentially requiring a shift to alternative consensus mechanisms
like Proof of
Stake (PoS) or consortium-based approaches.

\subsection{Validation}
We previously said that, because blockchain allows updates to be transparent to all participants, no trust is
required as updates can be validated independently.
This is true if we assume that we are able to integrate validation in the consensus mechanism; however this
is not trivial to achieve.

Suppose that nodes perform independent validation on the updates present in newly created blocks.
This would break the consensus mechanism: some nodes may consider a block valid whereas other nodes may not.

We would get to the same conclusion even if we assigned validation to a subset of nodes (e.g the ones
responsible for block creation).
In order for validation to be included in the consensus mechanism it needs to be deterministic and consistent
on all nodes.
However this introduces a new problem: a malicious actor that is able to see the validation dataset would be
able to generate a
backdoor in the model \cite{backdoorFL}.
This problem can be addressed by using a DON (Decentralized Oracle Network) that can queried by the nodes to
get a "yes" or "no" answer on the validity of the update \cite{Prof}.
This however raises the question of who should provide the validation dataset and the trustworthiness of the DON.

\section{Discussion of implementation choices}\label{sec:bfl-implementation}
Before moving on to our implementation \ref{chap:implementation} we would like to discuss pros and cons of
different implementation choices. Refer to \ref{sec:blockchain-classification} for an explaination of the
technical terms.

\subsection{Consensus Mechanism} Proof of Work as consensus mechanism has been used in works such as BlockFL
\cite{BlockFL}.
However PoW is not a good fit for our use case as it is very energy consuming and has a higher risks of
forking compared to other consensus mechanisms.
Proof of stake is more promising because it is energy efficient which is a strong requirement when dealing
with low-power devices such as the ones participating in federated learning.
Consortium-based consensus mechanisms are more centralized but are the most natural choice if our federated
learning task is private or multi-layer.
For instance in the case of a task distributed across hospitals, the hospitals could be the members of the consortium.

\subsection{Public or Private}
Let's first consider the fact that implementing a robust blockchain is by no means an easy feat and still a
field of ongoing research. Leveraging a public blockchain for federated learning tasks would allow to reuse
existing and battle-tested technology. Furthermore public blockchains are more decentralized and secure than
private chains due to their higher level of participation.
However their cost may be prohibitive for this use case.
Let's take ethereum as an example: the cost of storing 256 bits on chain is 20,000 gas according to the white
paper \cite{Ethereum}. This makes the cost of a kilobyte worth of storage 640,000 gas, which is about 20 USD
at the time of writing.
This makes storing weights on a public blockchain unfeasible.

Another aspect to consider is access-control: public blockchains do not offer access control mechanisms out of the box.
Although, smart contracts could be used to implement application-layer access control, data would still be
stored on chain and be accessible to everyone.
If the federated learning task is private then a private blockchain is the most natural choice.

\subsection{Using smart contracts} Smart contracts could be used to implement the logic of federated learning
directly on chain. This includes updating the model, validating the updates, rewarding the nodes and so on.
Nowadays, the existing frameworks for creating private blockchains, such as Cosmos or Hyperledger Fabric,
come with smart contract support.
However, following an approach similar to that of bitcoin \ref{sec:bitcoin} with limited or no scripting
capabilities at all could be more lightweight. This is especially considering that a private chain reserved
to federated learning would not need the power of smart contracts.
As a matter of fact the implementation presented in this document is not based on smart contracts, as we'll
see in \ref{chap:implementation}.

