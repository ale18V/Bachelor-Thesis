\section{Proposed implementation}\label{chap:implementation}
In \ref{sec:blockchain-fl} we discussed extensively about the pros and cons of blockchain and federated
learning integration.
We also mentioned the main challenges that arise when trying to combine these two technologies. We should be
explicit on the fact that it is a non-goal of this project to
deal with all, if any of them. The main goal of this project is to provide a working implementation of a
federated learning combined with blockchain and compare the performance of it with client-server vanilla FL.
However this does not mean that no effort has been put into trying to develop a robust solution.
In particular, great care has been put into the design of the system to improve upon designs suggested in
papers such as \cite{FlwrBC}, \cite{BlockFL}, \cite{VBFL}.
In facts:
\begin{enumerate}
  \item The model proposed in \cite{FlwrBC} uses the ethereum blockchain to implement a reward system.
    As mentioned in \ref{sec:bfl-implementation} storing the model and updates on a public chain would be
    unreasonable, thus the models are cleverly stored on IPFS instead of on chain. However this makes
    aggregating the updates on chain impossible so it still relied on a central server incurring in all
    problems related to centralization.
  \item The model proposed in \cite{BlockFL} defines a custom blockchain specific to federated learning. This
    is the same path that will be taken in this project. However the article based the architecture on PoW consensus,
    which is not suitable for a FL scenario as we argued in \ref{sec:bfl-implementation}.
  \item The model proposed in \cite{VBFL} is very similar to the one proposed in this project. It is
    basically a more featureful version of \cite{BlockFL}: it integrates a reputation system in the consensus
    algorithm to minimize malicious updates to the model. The main problem is that it still relies on a
    variation of PoW.

\end{enumerate}

The implementation presented here was thought to be a more efficient implementation of the last one of those articles.
It still includes a reputation system but is based on a pure proof of stake consensus which substitutes the
resource expensive PoW.
To sum up, the main goals of this implementation are:
\begin{enumerate}
  \item To implement a fairly robust blockchain network specialized for a federated learning task.
  \item To equip the blockchain network with a mechanism for validating updates so as to minimize the
    impact of malicious nodes.
  \item To make the blockchain operations as lightweight as possible, in order to be friendly to low-power devices.
\end{enumerate}

\subsection{Network layer}
The network protocol was heavily inspired by Bitcoin's, which we discussed in \ref{sec:bitcoin-network}.
For the sake of the experiment we left out the version exchange messages leaving the connection to be started
by directly sharing known peers.
The moment a node comes online it asks known peers to share the blockchain state so that it can synchronize
with the network.
As common in peer to peer networks we follow the \textit{gossip} protocol to share information about the
blockchain: pending transactions and blocks are all broadcasted to all known peers.
Akin to bitcoin we implemented persistent connections and a ping/pong messaging system to detect unhealthy peers.

\subsection{Consensus layer}\label{sec:consensus}
As far as the consensus algorithm is concerned, we chose to implement the Tendermint algorithm.
This is because the algorithm is robust and lays on strong mathematical foundations \cite{Tendermint}.
In particular, it is a byzantine fault tolerant algorithm, meaning that it can guarantee correct proceeding
of the operations of the network even if some malicious nodes are present.

We want to give a basic understanding of how the algorithm works. First thing first let's deine the network model.
Suppose that each node of the network has a voting power proportional to its stake or reputation and the
total voting power of the network is $n$.
Let $f$ be the number such that $n = 3f + 1$, then the algorithm works as follows:

\begin{enumerate}
  \item \textbf{Proposal message}: A node is elected as block proposer and proposes a block; a round is started.
  \item \textbf{Prevote message}: Nodes receive the block and cast votes in favour or against it depending on
    whether they consider it as valid.
  \item \textbf{Precommit message}: When a node receives at least $2f/3$ prevotes in favour of the block it
    sends a precommit message and locks the chosen block.
    It cannot vote for any other blocks in the same round.
  \item \textbf{Commit}: When a node receives at least $2f/3$ precommits in favour of a block it adds it to the chain.
  \item \textbf{New round}: If a block is not committed in a round a new round is started.
\end{enumerate}

The algorithm also implements timeouts in order not to incur into deadlocks. It is guaranteed that if at
least $2f + 1$ nodes are honest the algorithm will always reach consensus.

We refer the reader to \cite{Tendermint} if they're interested in a more technical explaination of the
protocol and the proof of its correctness.

\subsection{Blockchain layer}
The blockchain layer is the core of the system. It is where the model updates and the reputation of the nodes
are stored. The reputation of a node is determined by the amount of chain-native tokens it owns. These tokens
can be seen as a form of currency. This currency is minted after successful updates of the model.

The general overviow of the application-layer flow is the following:
\begin{itemize}
  \item \textbf{Model update request}: A node trains a model on its local dataset and wants to share this
    update. It crafts a transaction
    containing the update and broadcasts it to the network. This transaction is called an \textbf{update transaction}.
  \item \textbf{Validation}: This transaction is stored inside the mempool of a validator. The validator
    tests the update to see
    whether it is malicious or low quality. If this is the case the update is discarded.
  \item \textbf{Block creation and reward}: When one of the validators who received the update is chosen as
    proposer, it gathers the transaction
    along with all the other valid transactions inside a block. A \textbf{coinbase transaction} is also added
    to it, in order to reward all the contributors.
  \item \textbf{Consensus}: The block is then broadcasted to the network. The nodes of the chain will then
    reach agreement on
    whether to accept the block or not based on the consensus algorithm \cite{Tendermint}.
\end{itemize}

Let's dive into more details about the choises now.
\subsubsection{Reward}
Rewarding nodes with tokens provides an incentive mechanism to encourage participation in the network.
This is one of the lackings of Vanilla FL as we discussed in \ref{sec:challenges-vanilla-fl} and we try to
overcome it in this way in our implementation. Ideally we would want to reward nodes based on the quality of
their update. For simplicity we suppose that the rewards are fixed in our system.

In facts it is not trivial to determine the quality of an update: in contrast to centralized machine
learning, federated learning participants dont't share a global testing dataset. Thus quality would have to
be judged using local datasets.

In this scenario a validator node may judge an update as very high quality whereas another one may judge it
as barely acceptable.
This would lead to consensus issues, because the former would decide to reward generously the contributor, whereas the
latter would not. We cannot even decide to just leave the responsibility of judging the quality of the update
to the single proposer, because this introduces bias in the reward system.

\paragraph{Security}
All transactions contain the public key (address) of the creator node and a signature to prove the ownership
of the update (see \ref{sec:messages}).
Participants of the network can be addressed by their public key, similarly to bitcoin
\ref{table:bitcoin_output_format}.
This guarantees integrity of messages exchanged with peers in the network and is essential to build a reputation system.

Although this is a good start there is still a major flow in the proposed architecture. Signatures don't
provide confidentiality and everyone can read signed data and re-sign it as theirs.
This is a problem because a peer node can perform a \textbf{man-in-the-middle} attack on update transactions
and steal the
update by signing it with its own key. This is a problem that can be solved with a more sophisticated protocol or/and
using more advanced cryptography such as commitments or homomorphic encryption. For the sake of this
experiment we ignore this problem as we are not concerned with network-level attacks.

\paragraph{Validators}
The main idea behind having validator nodes is to provide a mechanism against malicious updates.
As we stated already in \ref{subsec:challenges-blockchain-fl}, in a blockchain based on PoW, we cannot
implement a network-wide validation mechanism, because asking every node to validate blocks with their own
independent dataset would break the consensus algorithm. On the other hand if we were to delegate the
validation to miners only, then  powerful attackers could easily craft malicious blocks.

The solution to this problem is to incorporate a voting mechanism in the consensus algorithm. This is what
has been done by choosing the Tendermint algorithm. If at least 2/3 of the votes of the network
are casted in favour of a proposed block then it is valid.

However, we don't want all nodes of the network to be held accountable for validation.
This is because being a validator requires:
\begin{enumerate}
  \item Having an high quality validation dataset.
  \item Having enough computational power to frequently validate transactions.
  \item Being frequently available, otherwise the network would be stuck.
\end{enumerate}
The majority of federated learning devices doesn't satisfy all three of those requirements.
Incidentally, restricting the number of validators contributes to making network convergence faster and is
more communication efficient.

If restricting the validators to a network subset we must also take care of prevenging malicious nodes from
becoming validators however.
This problem can be solved by implementing a reputation system based on the amount of tokens held by nodes.
If we assume that malicious nodes will be penalized and not rewarded well over the course of the FL task then
only honest nodes will own the amount of required tokens to become validators.

Note that restricting the number of validators also secures the network against \textit{sybil attacks} (a
  kind of attack where a malicious actor creates a large number of identities to gain
control of the network.)

\paragraph{Staking}
In order to preventing validators from disrupting the network we require them to stake funds.
This successfully prevents \textit{nothing at stake attacks}: ideally validators that misbehave should be punished by
burning their stake, either entirely or partially. However, for the purpose of the experiment we did not implement it.

In our network, a staking event is just a transaction where a node asks to stake some of its funds via a
\textbf{stake transaction}.

\paragraph{Block proposers}
In order to prevent chaos and risks of forking only one amongst the set of validators is chosen to propose a
block in a given round.
This is also mandated by the consensus algorithm that we chose \ref{sec:consensus}.
In our network the block propser is chosen with a deterministic random algorithm.
This is done via seeding the RNG using the current block height and round number. \ref{lst:random-proposer}
\begin{listing}\caption{Random proposer selection}\label{lst:random-proposer}
  \begin{minted}{python3}
def get_proposer(height: int, round: int, validators: set[bytes]):
"""Returns the proposer for the given height and round."""
    if not validators:
        return None
    random.seed(height + round)
    return random.choice(sorted(validators))

  \end{minted}
\end{listing}

\subsection{Technology stack}
The implementation of the project was primarily carried out in Python, which is the predominant language in
the field of machine learning. Python offers an extensive ecosystem of high-quality libraries, making it an
ideal choice for tasks related to both federated learning and blockchain integration. For the machine
learning component, we utilized PyTorch, a versatile and widely-adopted library that simplifies model
development and training.

For federated learning, we leveraged Flower, a Python library specifically designed to facilitate the
implementation of traditional federated learning systems. While Flower excels in client-server-based
architectures, integrating it with a blockchain network required modifications to its configuration. These
adjustments were necessary to adapt the framework to a decentralized, peer-to-peer architecture.

On the blockchain side, the network layer initially employed WebSocket communication. However, the
asynchronous nature of WebSocket messaging presented challenges in ensuring reliable and consistent
communication at the application level. To address these issues, we transitioned to using gRPC (Google Remote
Procedure Call), a language-agnostic RPC framework that runs on top of HTTP/2.0. gRPC allows for efficient,
bi-directional streaming of messages, with message formats defined using the Protocol Buffers syntax.

While gRPC inherently follows a client-server paradigm, this limitation was overcome by designing each node
to function as both a client and a server. This dual-role approach enabled seamless peer-to-peer
communication while retaining the reliability and performance benefits of gRPC.

\subsection{Messages and Data structures}\label{sec:messages}
Here are the common data structures used in the network and blockchain layers:
\begin{minted}[fontsize=\scriptsize]{proto}
/** Peer addresses in form of host:port */
message NetworkAddress {
    string address = 1;
}

/** Block format */
message Block {
    BlockHeader header = 1;
    BlockBody body = 2;
}

/** Block header contains metadata */
message BlockHeader {
    uint64 height = 1;
    uint64 round = 4;
    uint64 timestamp = 2;
    bytes parent = 3;
    bytes hash = 7;
}

/** Block body is a list of transactions */
message BlockBody {
    repeated Transaction transactions = 1;
}

/** General transaction format */
message Transaction {
    bytes public_key = 1;
    bytes signature = 2;
    uint32 timestamp = 3;
    TransactionType type = 4;
    TransactionData data = 5;
}

message TransactionData {
    oneof body {
    CoinbaseTransaction coinbase = 6;
    StakeTransaction vote = 8;
    UpdateTransaction update = 9;
    }
}

enum TransactionType {
    UPDATE = 1;
    COINBASE = 2;
    STAKE = 3;
}

/** A model update transaction */
message UpdateTransaction {
    bytes block = 2;
    string data = 1;
}

message CoinbaseTransaction {
    bytes address = 1;
    uint64 quantity = 2;
}

message StakeTransaction {
    bytes address = 1;
    uint64 quantity = 2;
}

message StakeAddress {
    bytes address = 1;
    uint64 quantity = 2;
}
\end{minted}

This is the RPC service definition:

\begin{minted}[fontsize=\scriptsize]{proto}
syntax = "proto3";
service Node {
  rpc AdvertisePeer(NetworkAddress) returns (google.protobuf.Empty) {};
  rpc RequestPeers(google.protobuf.Empty) returns (RequestPeersResponse) {};
  rpc AdvertiseTransaction(Transaction) returns (google.protobuf.Empty) {};
  rpc ProposeBlock(ProposeBlockRequest) returns (google.protobuf.Empty) {};
  rpc RequestBlock(BlockRequest) returns (BlockResponse) {};
  rpc AdvertisePrevote(PrevoteMessage) returns (google.protobuf.Empty) {};
  rpc AdvertisePrecommit(PrecommitMessage) returns (google.protobuf.Empty) {};
  rpc RequestBlockchain(google.protobuf.Empty) returns (BlockchainMessage) {};
  rpc RequestBalance(BalanceRequest) returns (BalanceResponse) {};
  rpc Ping(google.protobuf.Empty) returns (google.protobuf.Empty) {};
}

message RequestPeersResponse {
  repeated NetworkAddress addresses = 1;
}

message BlockchainMessage {
  repeated Block blocks = 1;
}

message BalanceRequest {
  optional bytes address = 1;
}

message BalanceResponse {
  optional uint64 balance = 1;
}

message ProposeBlockRequest {
  Block block = 1;
  bool is_previous = 5;
  bytes pubkey = 2;
  bytes signature = 3;
}
message PrevoteMessage {
  optional bytes hash = 1;
  uint64 height = 5;
  uint64 round = 6;
  bytes pubkey = 2;
  bytes signature = 3;
}

message PrecommitMessage {
  optional bytes hash = 1;
  uint64 height = 5;
  uint64 round = 6;
  bytes pubkey = 2;
  bytes signature = 3;
}
message BlockRequest {
  bytes hash = 1;
}

message BlockResponse {
  optional Block block = 1;
}
\end{minted}
