\subsection{Context of the Thesis}\label{chap:contextandobjectives}
Advancements in technology have driven a data-centric shift across industries such as healthcare, finance,
smart cities, and the Internet of Things (IoT). This growing reliance on data has introduced significant
challenges in privacy, security, and scalability, necessitating innovative solutions. This thesis explores
the intersection of two emerging paradigms—Federated Learning (FL) and Blockchain technology—as a potential
means to address these challenges.

\textbf{Federated Learning} has emerged as a transformative approach to machine learning by enabling
collaborative model training without the need to share raw data. This is achieved by keeping relevant
computation local to devices and only sharing updates with the interested party.
This preserves data privacy while still achieving high-performance learning. Despite its benefits, FL faces
critical challenges, including reliance on a central server, susceptibility to malicious participants, and a
lack of transparency in the training process.

\textbf{Blockchain technology}, initially developed as the backbone for cryptocurrencies like Bitcoin,
provides a decentralized, secure, and transparent framework for managing data and operations among untrusted
parties. With its core features—such as consensus mechanisms, immutability, and smart contracts—Blockchain
has demonstrated value in sectors ranging from finance to supply chain management. However, its scalability
limitations and resource-intensive operations can hinder its applicability in dynamic environments.

Integrating FL and Blockchain presents a promising avenue for overcoming the limitations of both
technologies. Blockchain's decentralization can eliminate the dependency on a central server in FL, while its
transparency and consensus protocols enhance trust and accountability. Conversely, FL's distributed learning
capabilities reduce the need for raw data storage on the Blockchain, mitigating its scalability concerns.

This thesis is motivated by real-world challenges such as:
\begin{itemize}
  \item Ensuring privacy and security in data-sharing across regulated sectors like healthcare and finance.
  \item Establishing trust in collaborative environments where participants may not fully trust each other,
    such as in decentralized IoT systems.
\end{itemize}

While Federated Learning and Blockchain are an ongoing field of research and a fair amount of articles have
been published about these topics,
their integration has been underexplored. Combining the two technologies introduces novel challenges
including communication privacy and efficiency, synchronism of updates, security and scalability.

Through the design and evaluation of a Blockchain-enhanced Federated Learning framework, this thesis aims to
provide a foundational understanding of the two technologies and explore the challanges and opportunities in
context of privacy, trust, and scalability. The following section outlines the specific objectives of this
thesis, which are derived from this contextual foundation.

\subsection{Objectives of the Thesis}

This document aims to explore the feasibility and practicality of implementing a federated learning system on
a blockchain network. The contributions of this work are threefold:

\paragraph{Explore the technologies}
The thesis seeks to provide to the reader a foundational understanding of both Federated Learning and Blockchain
technologies. We will try to be as thorough as possible, within the limits of the purposes of this document,
highlighting the individual strengths, weaknesses, and potential areas of
application of both technologies.

\paragraph{Exploring the integration}
After providing an understanding of the technologies and their strengths and weaknesses we want to discuss
the advantages and challenges that arise when joining them.

We will discuss them in depth in the coming chapters but as an anticipation, some of the advantages include:
\begin{itemize}
  \item Addressing the central server reliance in FL through Blockchain's decentralized
    infrastructure.
  \item Enhancing trust, transparency, and security in FL systems using Blockchain's
    consensus mechanisms and immutability.
  \item Provide incentives for devices to partake in the federated learning process.
\end{itemize}

And some of the challenges include:
\begin{itemize}
  \item Computational and communication overhead due to Blockchain protocols.
  \item Ensuring security and consistence of the gloabl model.
\end{itemize}

When discussing the challenges and advantages we will also take a look at existing
research and how some of these problems are dealt with and how some advantages are successfully exploited.

\paragraph{Proposing our own implementation}
After listing implementative and theoretical challenges we will proceed to propose our own implementation
of a federated learning system on a blockchain network.

We will try to compare it with the research papers discussed over the course of the document and provide our
own unique opinions and implementation choices ragarding topics with:
\begin{itemize}
  \item Outlining the network, consensus, and Blockchain layers to integrate the two
    technologies effectively.
  \item Exploring implementation choices, such as public versus private Blockchains,
    consensus mechanisms, and the role of smart contracts.
  \item Dealing with malicious updates in a robust manner.
  \item Making the system communication efficient and address storage and availability challenges.
\end{itemize}

We will then proceed to evaluate our implementation and compare it to a classic client-server implementation
and show the feasibility of blockchain integration.

\paragraph{Contributing to Academic research}
Saying that one of the aims of this document is to contribute to academic research may sound a bit pretentious
as this is by no means a groundbreaking work. However, by gathering existing research and providing our own
implementation and opinions on some relevant topics we hope that this document will be useful to someone
looking for a starting point in this field.

\paragraph{Conclusion}
By achieving these objectives, the thesis endeavors to shed light on the transformative
potential of combining Blockchain and Federated Learning, laying the groundwork for innovative
solutions to contemporary challenges in privacy-preserving and secure collaborative learning.
