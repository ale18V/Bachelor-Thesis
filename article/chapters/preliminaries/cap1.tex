\section*{Titolo cap}\label{chap:introduction}

In recent years, advancements in distributed computing and decentralized technologies have reshaped the
landscape of machine learning and data processing. This document explores the intersection of two
groundbreaking paradigms: Federated Learning (FL) and Blockchain. Each of these fields has garnered
significant attention for its unique potential — federated learning for enabling collaborative model training
without centralized data aggregation, and blockchain for providing decentralized governance and immutable ledgers
with robust security guarantees.

\section{Federated Learning and Blockchain}

Federated learning is a decentralized approach to training machine learning models collaboratively across
multiple devices or nodes while keeping data local to each participant. Originally developed to address
privacy concerns, FL has found applications in domains such as healthcare, mobile devices, and IoT, where
sensitive data remains confined to the edge devices that generate it. However, FL is not without its
challenges. We will give a brief introduction to machine learning in \ref{chap:machine-learning}
as a preliminary before introducing federated learning in \ref{chap:federated-learning}.

Blockchain, on the other hand, is a decentralized ledger technology that ensures secure and immutable
decentralized data storage and trustless interactions among participants.
First popularized by cryptocurrencies such as Bitcoin, blockchain has since evolved to support a wide range
of use cases, including supply chain management, smart
contracts, and secure data sharing. We will delve into the fundamentals of this technology in chapter \ref{chap:blockchain}.

\section{Combining Federated Learning and Blockchain}

The possibility of integrating blockchain with federated learning has sparked interest in the research community as a
means to address some of the inherent limitations of the paradigm.
Federated learning's reliance on a central server for model aggregation introduces risks of a single point of failure
and trust issues among participants.
Blockchain's decentralized nature and consensus mechanisms can solve this problem among others, such as
ensuring model integrity and fairness in participation. However this comes with its own set of challenges,
mostly communication, storage and (depending on the underlying mechanism of the blockchain, also computation) overhead.
We will delve into this topic in \ref{sec:blockchain-fl}

Integrating blockchain into federated learning systems presents unique challenges. These include the
computational and communication overhead introduced by blockchain protocols, ensuring scalability in systems
with millions of devices, and aligning the asynchronous nature of federated learning with blockchain's
consensus requirements. Despite these hurdles, the potential benefits—such as enhanced security,
transparency, and trust—make this integration a promising avenue of research.

\section{Contributions and Objectives}

This document aims to explore the feasibility and practicality of implementing a federated learning system on
a blockchain network. The contributions of this work are threefold:

\begin{itemize}
  \item \textbf{Conceptual Exploration:} We investigate the theoretical basis and potential synergies between
    federated learning and blockchain technologies.
  \item \textbf{Implementation:} We present an implementation of a federated learning system underpinned by
    blockchain, addressing key challenges such as communication efficiency, scalability, and security.
  \item \textbf{Comparative Analysis:} We compare our implementation with existing solutions from the
    literature, highlighting its strengths and identifying areas for future improvement.
\end{itemize}

Through these contributions we aim to provide an understanding of how and why federated learning could be
integrated with blockchain technology and highlight potential challanges.

\section{Structure of the Document}

The remainder of this document is organized as follows:
\begin{itemize}
  \item Chapter \ref{chap:blockchain} delves into the fundamentals of blockchain technology, examining its
    mechanisms, benefits, and limitations, focusing on the Bitcoin network as a case study.

  \item Chapter \ref{chap:machine-learning} is a very basic introduction to machine learning, which
    introduces some topics and terminology relevant to federated learning.
  \item Chapter \ref{chap:federated-learning} introduces the principles of federated learning, its
    architecture, and its applications. Specifically, in \ref{sec:vanilla-fl} we explore the original
    client-server model, whereas in \ref{sec:blockchain-fl} we explore blockchain integration.
  \item Chapter \ref{chap:implementation} presents the proposed implementation of a federated learning system
    on a blockchain network.
  \item Chapter \ref{chap:evaluation} compares our implementation to existing solutions, discussing its
    effectiveness and scalability.
  \item Finally, Chapter \ref{chap:conclusion} concludes the document, summarizing findings and proposing
    directions for future research.
\end{itemize}
